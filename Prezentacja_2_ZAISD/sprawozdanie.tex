\documentclass[UTF8]{article}

\usepackage{amsmath}
\usepackage[utf8]{inputenc}
\usepackage[OT4]{polski}
\usepackage{graphicx}
\usepackage{subcaption}
\usepackage{geometry}
\usepackage{titlesec}
\usepackage{graphicx}

\newcommand{\var}{\texttt}
\newcommand{\assign}{\leftarrow}

\usepackage{listings}
\usepackage{color}

\usepackage{capt-of}


\definecolor{dkgreen}{rgb}{0,0.6,0}
\definecolor{gray}{rgb}{0.5,0.5,0.5}
\definecolor{mauve}{rgb}{0.58,0,0.82}

\lstset{frame=tb,
	language=c++,
	aboveskip=3mm,
	belowskip=3mm,
	showstringspaces=false,
	columns=flexible,
	basicstyle={\small\ttfamily},
	numbers=none,
	numberstyle=\tiny\color{gray},
	keywordstyle=\color{blue},
	commentstyle=\color{dkgreen},
	stringstyle=\color{mauve},
	breaklines=true,
	breakatwhitespace=true,
	tabsize=3
}

\graphicspath{ {pics/} }

\geometry{left=2.5cm,right=2.5cm,top=2.5cm,bottom=2.5cm}
\titlelabel{\thetitle.\quad}

\title{%
	Sprawozdanie 2 \\~
	\large\flushleft Zadanie 10. Liczbę naturalną $C$ można przedstawić jako sumę parami różnych liczb naturalnych. Na przykład jeśli $C = 6$, to możemy C przedstawić na cztery sposoby:~
	\\$1+2+3$
	\\$1+5$
	\\$2+4$
	\\$6$
	\\a jeśli $C = 10$, to takimi podziałami są:~
	\\$1+2+3+4$
	\\$1+2+7$
	\\$1+3+6$
	\\$1+4+5$
	\\$1+9$
	\\$2+3+5$
	\\$2+8$
	\\$3+7$
	\\$4+6$
	\\$10$
	\\Skonstruuj algorytm wyczerpujący z nawrotami, generujący wszystkie podziały podanej liczby naturalnej $C$.~}

\author{Adrian Rupala}


\begin{document}
\pagenumbering{gobble}
\maketitle

\newpage
\tableofcontents

\newpage
\pagenumbering{arabic}

\section{Teoria} ~

Algorytm z nawrotami to algorytm wyszukiwania wszystkich lub kilku rozwiązań. Polega on na znajdowaniu wyniku metodą „prób i błędów”, wszelako z oznaczeniem niepowodzeń, dzięki czemu te same błędy nie są popełniane dwukrotnie. ~\\

Jeżeli problem pozwala na zastosowanie algorytmu wyszukiwania z nawrotami, to metoda ta może być znaczenie efektywniejsza niż wyszukiwanie wyczerpujące (zakładające przeszukiwanie wszystkich rozwiązań), ponieważ pojedynczy test może wyeliminować nie jedno, a wiele rozwiązań niedopuszczalnych. ~\\

Rekurencja to technika programowania, dzięki której funkcja, procedura lub podprogram jest w stanie w swoim ciele wywołać sama siebie. Pozwala ona łatwo wykonać wiele zadań, w których zachodzi potrzeba obliczenia wyników cząstkowych do obliczenia całości. ~

\section{Rozwiązanie}~

Oto pseudokod przedstawiający rozwiązanie problemu przedstawionego w zadaniu.

\begin{lstlisting}

bool znajdz_duplikat(int tablica[], int rozmiar_tablicy) {
	sort(tablica);
	for (int i = 1; i < rozmiar_tablicy - 1; i++) {
		if (tablica[i] == tablica[i + 1]){
			return true;
		}
	}
	return false;
}	

void sprawdz_i_wypisz(int pozycja, int pozostalo) {
	if (pozostalo == 0) {
		for (int i = 1; i <= pozycja - 1; i++) {
			cout << tablica[i] << " + ";
		}
		cout << endl;
	} else {
		if(znajdzDuplikat(tablica, pozycja) == false){
			for (int k = tablica[pozycja - 1]; k <= pozostalo; k++) {
				tablica[pozycja] = k;
				sprawdz_i_wypisz(pozycja + 1, pozostalo - k);
			}
		}
	}
}	

void wywolanie(int C) {
	tablica[0] = 1;
	sprawdz_i_wypisz(1, C);
}
		
\end{lstlisting}
\newpage
Na samym początku przedstawiona została funkcja \texttt{znajdz\char`_duplikat}. Sortuje ona tablicę, a następnie porównuje wszystkie istniejące elementy, aby zobaczyć czy występują powtórzone wartości. Zwraca ona logiczną wartość \texttt{true} jeśli występuje powtórzenie lub \texttt{false} jeśli porównywane wartości są różne. ~\\

Kolejna funkcja, nazwana  \texttt{sprawdz\char`_i\char`_wypisz}, odpowiada za generowanie ciągów liczbowych i wypisywanie ich. Najpierw funkcja sprawdza ilość pozostałych wartości ciągu liczbowego. Jeżeli ten parametr ma wartość 0, zawartość ciągu zostaje wypisana. W przeciwnym wypadku sprawdzamy, czy pojedynczy element w tablicy wystąpił już wcześniej. Jeżeli nie znaleziono identycznego elementu, algorytm iteruje po kolejnych elementach ciągu liczbowego tak długo, jak ilość pozostałych elementów jest większa bądź równa zero. W momencie spełnienia warunku, funkcja wywoływana jest rekurencyjnie, gdzie parametr pozycji zostaje zwiększony o jeden, a ilość poszukiwanych liczb jest zmniejszana.~\\

Ostatnia funkcja algorytmu odpowiada za przypisanie pierwszej wartości tablicy liczby 1, ponieważ jest to najmniejsza wartość, jaką może osiągnąć liczba w naszym rozkładzie oraz zostaje wywołana funkcja \texttt{sprawdz\char`_i\char`_wypisz} z parametrami: 1 jako pierwsza wartość pozycji, oraz liczbą, jaka ma zostać rozłożona.~\\

\end{document}