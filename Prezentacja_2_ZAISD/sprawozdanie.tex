\documentclass[UTF8]{article}

\usepackage{amsmath}
\usepackage[utf8]{inputenc}
\usepackage[OT4]{polski}
\usepackage{graphicx}
\usepackage{subcaption}
\usepackage{geometry}
\usepackage{titlesec}
\usepackage{graphicx}

\newcommand{\var}{\texttt}
\newcommand{\assign}{\leftarrow}

\usepackage{listings}
\usepackage{color}

\usepackage{capt-of}


\definecolor{dkgreen}{rgb}{0,0.6,0}
\definecolor{gray}{rgb}{0.5,0.5,0.5}
\definecolor{mauve}{rgb}{0.58,0,0.82}

\lstset{frame=tb,
	language=c++,
	aboveskip=3mm,
	belowskip=3mm,
	showstringspaces=false,
	columns=flexible,
	basicstyle={\small\ttfamily},
	numbers=none,
	numberstyle=\tiny\color{gray},
	keywordstyle=\color{blue},
	commentstyle=\color{dkgreen},
	stringstyle=\color{mauve},
	breaklines=true,
	breakatwhitespace=true,
	tabsize=3
}

\graphicspath{ {pics/} }

\geometry{left=2.5cm,right=2.5cm,top=2.5cm,bottom=2.5cm}
\titlelabel{\thetitle.\quad}

\title{%
	Sprawozdanie 2 \\~
	\large\flushleft Zadanie 10. Liczbę naturalną $C$ można przedstawić jako sumę parami różnych liczb naturalnych. Na przykład jeśli $C = 6$, to możemy C przedstawić na cztery sposoby:~
	\\$1+2+3$
	\\$1+5$
	\\$2+4$
	\\$6$
	\\a jeśli $C = 10$, to takimi podziałami są:~
	\\$1+2+3+4$
	\\$1+2+7$
	\\$1+3+6$
	\\$1+4+5$
	\\$1+9$
	\\$2+3+5$
	\\$2+8$
	\\$3+7$
	\\$4+6$
	\\$10$
	\\Skonstruuj algorytm wyczerpujący z nawrotami, generujący wszystkie podziały podanej liczby naturalnej $C$.~}

\author{Adrian Rupala}


\begin{document}
\pagenumbering{gobble}
\maketitle

\newpage
\tableofcontents

\newpage
\pagenumbering{arabic}

\section{Teoria} ~

Algorytm z nawrotami to algorytm wyszukiwania wszystkich lub kilku rozwiązań. Polega on na znajdowaniu wyniku metodą „prób i błędów”, wszelako z oznaczeniem niepowodzeń, dzięki czemu te same błędy nie są popełniane dwukrotnie. ~\\

Jeżeli problem pozwala na zastosowanie algorytmu wyszukiwania z nawrotami, to metoda ta może być znaczenie efektywniejsza niż wyszukiwanie wyczerpujące (zakładające przeszukiwanie wszystkich rozwiązań), ponieważ pojedynczy test może wyeliminować nie jedno, a wiele rozwiązań niedopuszczalnych. ~\\

Rekurencja to technika programowania, dzięki której funkcja, procedura lub podprogram jest w stanie w swoim ciele wywołać samą siebie. Pozwala ona łatwo wykonać wiele zadań, w których zachodzi potrzeba obliczenia wyników cząstkowych do obliczenia całości. ~

\section{Rozwiązanie}~

Oto algorytm zaimplementowany w języku C++ przedstawiający rozwiązanie problemu przedstawionego w zadaniu:

\begin{lstlisting}

#include <iostream>

using namespace std;

int tablica[100];

void podzial(int C, int obecny, int* tablica, int index) {
	if (obecny + tablica[index] == C) {
		for (int i=0; i <= index; i++) {
			cout << tablica[i] << " ";
		}
		cout << endl;       
		;
	} else if (obecny + tablica[index] > C) {
		;
	} else {
		for(int i = tablica[index]+1; i < C; i++) {
			tablica[index+1] = i;
			podzial(C, obecny + tablica[index], tablica, index+1);
		}
	}
}

int main(){   
	int C = 0;

	cout << "Podaj liczbe: " << endl;
	cin >> C;
	cin.get();
	cout << "======" << endl;

	for(int i = 1; i <= C; i++) {
		tablica[0] = i;
		podzial(C, 0, tablica, 0);
	}

	cin.get();
	return 0;
}
		
\end{lstlisting}

Kod programu składa się z dwóch funkcji: \texttt{podzial} oraz głównej funkcji \texttt{main}. ~\\

Funkcja \texttt{main} ma za zadanie pobranie od użytkownika liczby, przekazanie jej do zmiennej \texttt{C} oraz wypisanie kolejnych tablic, które zawierają pożądane przez nas liczby wraz z rekurencyjnym wywołaniem funkcji \texttt{podzial}. ~\\

Funkcja \texttt{podzial} przyjmuje jako parametry liczbę którą chcemy rozłożyć na czynniki - \texttt{C}, obecnie porównywany element - \texttt{obecny}, wskaźnik na pierwszy element tablicy - \texttt{tablica}, oraz kolejny indeks tablicy - \texttt{index}. W pierwszej instrukcji warunkowej \texttt{if} następuje sprawdzenie, czy suma obecnego elementu, oraz elementu tablicy jest równa szukanej liczbie, jeśli tak, zawartość tablicy jest wypisywana. Sprawdzenie to ma na celu ustalenie, czy suma elementów rozkładu liczby przechowywana w tablicy osiągnęła liczbę, jaką chcemy rozłożyć. Kolejny warunek sprawdza, czy ta suma jest większa od liczby podanej przez użytkownika, jeśli tak, nie zwraca nic. W każdym innym przypadku, za pomocą pętli \texttt{for} zwiększamy kolejny element tablicy. Wtedy, gdy jest on mniejszy od podanej przez nas liczby, zwiększamy indeks tablicy o jeden, oraz wywołujemy rekurencyjnie całą funkcję ze zmienionymi parametrami zwiększonego o jeden indeksu, obecnego elementu jako sumy elementu obecnego, oraz aktualnego elementu tablicy.~

\end{document}