\documentclass[aspectratio=169]{beamer}
\usetheme{metropolis}

\usepackage[utf8]{inputenc}
\usepackage[OT4]{polski}
\usepackage{flexisym}
\usepackage{parskip}
\usepackage[]{algorithm2e}


\title{Zadanie 10. Podział liczby}
\subtitle{Adrian Rupala}


\begin{document}
	\frame {
		\titlepage
	}

	\frame {
		\frametitle{Treść zadania}
		\fontsize{12pt}{7.2}\selectfont
		Liczbę naturalną $C$ można przedstawić jako sumę parami różnych liczb naturalnych. Na przykład Jeśli $C = 6$, to możemy C przedstawić na cztery sposoby:
		\\$1+2+3$
		\\$1+5$
		\\$2+4$
		\\$6$
		\\a jeśli $C = 10$, to takimi podziałami są:
		\\$1+2+3+4$
		\\$1+2+7$
		\\$1+3+6$
		\\$1+4+5$
		\\$1+9$
		\\$2+3+5$
		\\$2+8$
		\\$3+7$
		\\$4+6$
		\\$10$
		\\Skonstruuj algorytm wyczerpujący z nawrotami, generujący wszystkie podziały podanej liczby naturalnej $C$.~
	}

	\frame{
		\frametitle{Definicje}
		\fontsize{14pt}{7.2}\selectfont
		\large Algorytm z nawrotami (backtracking) - algorytm wyszukiwania wszystkich lub kilku rozwiązań polegający na znajdowaniu wyniku metodą "prób i błędów", wszelako z oznaczeniem niepowodzeń, dzięki czemu te same błędy nie są popełniane dwukrotnie. ~ \bigskip
			
		Jeżeli problem pozwala na zastosowanie algorytmu wyszukiwania z nawrotami, to metoda ta może być znaczenie efektywniejsza niż wyszukiwanie wyczerpujące (zakładając przeszukiwanie wszystkich rozwiązań), ponieważ pojedynczy test może wyeliminować nie jedno a wiele rozwiązań niedopuszczalnych. ~
	}

	\frame{
		\frametitle{Rozwiązanie}
		\fontsize{13pt}{7.2}\selectfont	

	}
	
	\frame{
		\centering \Huge
		\emph{Dziękuję za uwagę!}
	}
	
\end{document}
