\documentclass[UTF8]{article}

\usepackage{amsmath}
\usepackage[utf8]{inputenc}
\usepackage[OT4]{polski}
\usepackage{graphicx}
\usepackage{subcaption}
\usepackage{geometry}
\usepackage{titlesec}
\usepackage{graphicx}

\newcommand{\var}{\texttt}
\newcommand{\assign}{\leftarrow}

\usepackage{listings}
\usepackage{color}

\usepackage{capt-of}


\definecolor{dkgreen}{rgb}{0,0.6,0}
\definecolor{gray}{rgb}{0.5,0.5,0.5}
\definecolor{mauve}{rgb}{0.58,0,0.82}

\lstset{frame=tb,
	language=c++,
	aboveskip=3mm,
	belowskip=3mm,
	showstringspaces=false,
	columns=flexible,
	basicstyle={\small\ttfamily},
	numbers=none,
	numberstyle=\tiny\color{gray},
	keywordstyle=\color{blue},
	commentstyle=\color{dkgreen},
	stringstyle=\color{mauve},
	breaklines=true,
	breakatwhitespace=true,
	tabsize=3
}

\graphicspath{ {pics/} }

\geometry{left=2.5cm,right=2.5cm,top=2.5cm,bottom=2.5cm}
\titlelabel{\thetitle.\quad}

\title{%
	Sprawozdanie 3 \\~
	\large\flushleft Pewien student rozwiązując test jednokrotnego wyboru, w którym każdemu pytaniu przyporządkowano pięć odpowiedzi (w tym dokładnie jedną prawidłową), zna poprawne odpowiedzi lub zgaduje. Niech prawdopodobieństwo tego, że zna poprawną odpowiedź na dane pytanie wynosi $\frac{1}{2}$. Czyli na połowę pytań odpowiada „strzelając”. Ponieważ jest pięć odpowiedzi do wyboru, szansa, że odgadnie prawidłową odpowiedź wynosi $\frac{1}{5}$. ~ \bigskip
	\\Jakie jest prawdopodobieństwo warunkowe tego, że student znał odpowiedź na dane pytanie, jeśli nie popełnił w tym pytaniu błędu? ~}	

\author{Adrian Rupala}


\begin{document}
\pagenumbering{gobble}
\maketitle

\newpage
\tableofcontents

\newpage
\pagenumbering{arabic}

\section{Teoria}

Celem zadania jest obliczenie prawdopodobieństwa warunkowego. Twierdzenie Bayesa jest bezpośrednio związane z obliczaniem prawdopodobieństwa warunkowego i ma ono na celu jego korygowanie na podstawie uzyskanych później informacje o zachodzących zdarzeniach. W tym zadaniu mamy również do czynienia~z~dwoma zdarzeniami zachodzącymi po sobie. Pierwsze z nich to prawidłowa odpowiedź na pytanie, kolejne to niepopełnienie błędu przez uczenia.~\\
\noindent \\W podstawowej formie twierdzenie Bayesa mówi, że:
\\$P(A_{i} | B) = \frac{P(B|A_{1})P(A_{i})}{P(B)}$, gdzie: $P(B) = \sum_{i \in j} P(B|A_{j}) \cdot P(A_{j}) $ \\
\\Kiedy $A$ i $B$ są zdarzeniami oraz $P(B) > 0$, przy czym:

\begin{itemize}
	\item $P(A|B)$ oznacza prawdopodobieństwo warunkowe (prawdopodobieństwo zajścia zdarzenia A o ile zajdzie zdarzenie B).~
	\item $P(B|A)$ oznacza prawdopodobieństwo zajścia zdarzenia B, o ile zajdzie zdarzenie A.~
\end{itemize}

\section{Rozwiązanie}

Z treści zadania można odczytać następujące dane:
\begin{itemize}
	\item $A_{1}$ - uczeń zna odpowiedź,
	\item $A_{2}$ -  uczeń nie zna odpowiedzi,
	\item $B$ - uczeń nie popełnił błędu.
\end{itemize}

\bigskip
\noindent Szukaną przez nas wartością jest prawdopodobieństwo zakładające, że uczeń zna odpowiedź, oraz, że uczeń nie popełnił błędu $P(A_{1}|B)$, następnie podstawiając do wzoru Bayesa otrzymujemy równanie:
\\$P(A_{1}|B) = \frac{P(B|A_{1}) \cdot P(A_{1})}{P(A_{1}) \cdot P(B|A_{1}) + P(A_{2}) \cdot P(B|A_{2})}$

\bigskip 
\noindent Z treści zadania wynika, że 
\begin{itemize}
	\item $P(A_{1}) = P(A_{2}) = \frac{1}{2}$, 
	\item $P(B|A_{1}) = 1$,
	\item $P(B|A_{2}) = \frac{1}{5}$.
\end{itemize}
Następnie podstawiając wartości do wzoru można otrzymać następujące równanie:\\
$P(A_{1}|B) = \frac{\frac{1}{2} \cdot 1}{\frac{1}{2} \cdot 1 + \frac{1}{2} \cdot \frac{1}{5}}$\\
\\Upraszczając równanie:
\\$P(A_{1}|B) = \frac{1}{2(\frac{\frac{1}{5}}{2} + \frac{1}{2})}$ $\Rightarrow$ $P(A_{1}|B) = \frac{1}{2(\frac{1}{10} + \frac{1}{2})}$ \\
\\Następnie można sprowadzić ułamki w mianowniku do wspólnego mianownika:
\\$P(A_{1}|B) = \frac{1}{2\cdot \frac{5+1}{10}}$ $\Rightarrow$ $P(A_{1}|B) = \frac{1}{2 \cdot \frac{6}{10}}$\\
\\*Kolejnym krokiem jest przeniesienie dwójki do licznika oraz uproszczenie postaci ułamka znajdującego się w mianowniku:
\\$P(A_{1}|B) = \frac{\frac{1}{2}}{\frac{3}{5}}$\\
\\*Mnożąc licznik przez odwrotność mianownika, otrzymamy równanie postaci:
\\$P(A_{1}|B) = \frac{5}{2 \cdot 3}$ $\Rightarrow$ $P(A_{1}|B) = \frac{5}{6}$

\newpage

\section{Odpowiedź}
Prawdopodobieństwo warunkowe tego, że student znał odpowiedź na pytanie, jeśli nie popełnił błędu, wynosi $\frac{5}{6}$.
\end{document}