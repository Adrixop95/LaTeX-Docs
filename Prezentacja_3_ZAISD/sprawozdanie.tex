\documentclass[UTF8]{article}

\usepackage{amsmath}
\usepackage[utf8]{inputenc}
\usepackage[OT4]{polski}
\usepackage{graphicx}
\usepackage{subcaption}
\usepackage{geometry}
\usepackage{titlesec}
\usepackage{graphicx}

\newcommand{\var}{\texttt}
\newcommand{\assign}{\leftarrow}

\usepackage{listings}
\usepackage{color}

\usepackage{capt-of}


\definecolor{dkgreen}{rgb}{0,0.6,0}
\definecolor{gray}{rgb}{0.5,0.5,0.5}
\definecolor{mauve}{rgb}{0.58,0,0.82}

\lstset{frame=tb,
	language=c++,
	aboveskip=3mm,
	belowskip=3mm,
	showstringspaces=false,
	columns=flexible,
	basicstyle={\small\ttfamily},
	numbers=none,
	numberstyle=\tiny\color{gray},
	keywordstyle=\color{blue},
	commentstyle=\color{dkgreen},
	stringstyle=\color{mauve},
	breaklines=true,
	breakatwhitespace=true,
	tabsize=3
}

\graphicspath{ {pics/} }

\geometry{left=2.5cm,right=2.5cm,top=2.5cm,bottom=2.5cm}
\titlelabel{\thetitle.\quad}

\title{%
	Sprawozdanie 3 \\~
	\large\flushleft Pewien student rozwiązując test jednokrotnego wyboru, w którym każdemu pytaniu przyporządkowano pięć odpowiedzi (w tym dokładnie jedną prawidłową), zna poprawne odpowiedzi lub zgaduje. Niech prawdopodobieństwo tego, że zna poprawną odpowiedź na dane pytanie wynosi $\frac{1}{2}$. Czyli na połowę pytań odpowiada „strzelając”. Ponieważ jest pięć odpowiedzi do wyboru, szansa, że odgadnie prawidłową odpowiedź wynosi $\frac{1}{5}$ ~ \bigskip
	\\Jakie jest prawdopodobieństwo warunkowe tego, że student znał odpowiedź na dane pytanie, jeśli nie popełnił w tym pytaniu błędu? ~}	

\author{Adrian Rupala}


\begin{document}
\pagenumbering{gobble}
\maketitle

\newpage
\tableofcontents

\newpage
\pagenumbering{arabic}

\section{Teoria} ~

Celem zadania jest obliczenie prawdopodobieństwa warunkowego. Twierdzenie Bayesa jest bezpośrednio związane z obliczaniem prawdopodobieństwa warunkowego i ma ono na celu jego korygowanie w oparciu o uzyskane później informacje o zachodzących zdarzeniach. W tym zadaniu mamy również do czynienia z dwoma zdarzeniami zachodzącymi po sobie. Pierwsze z nich to prawidłowa odpowiedź na pytanie, kolejne to nie popełnienie błędu przez uczenia.~\\

W podstawowej formie twierdzenie Bayesa mówi, że
\\$P(A | B) = \frac{P(B|A)P(A)}{P(B)}$,
\\gdzie $A$ i $B$ są zdarzeniami oraz $P(B) > 0$, przy czym

\begin{itemize}
	\item $P(A|B)$ oznacza prawdopodobieństwo warunkowe (prawdopodobieństwo zajścia zdarzenia A o ile zajdzie zdarzenie B).~
	\item $P(B|A)$ oznacza prawdopodobieństwo zajścia zdarzenia B o ile zajdzie zdarzenie A.~
\end{itemize}

\section{Rozwiązanie} ~



\newpage

\section{Odpowiedź} ~


\end{document}