\documentclass[aspectratio=169]{beamer}
\usetheme{metropolis}

\usepackage[utf8]{inputenc}
\usepackage[OT4]{polski}
\usepackage{flexisym}
\usepackage{parskip}

\title{Zadanie 45. Test}
\subtitle{Adrian Rupala}

\begin{document}
	\frame {
		\titlepage
	}

	\frame {
		\frametitle{Treść zadania}
		\fontsize{10pt}{7.2}\selectfont
		\large Pewien student rozwiązując test jednokrotnego wyboru, w którym każdemu pytaniu przyporządkowano pięć odpowiedzi (w tym dokładnie jedną prawidłową), zna poprawne odpowiedzi lub zgaduje. Niech prawdopodobieństwo tego, że zna poprawną odpowiedź na dane pytanie wynosi $\frac{1}{2}$. Czyli na połowę pytań odpowiada „strzelając”. Ponieważ jest pięć odpowiedzi do wyboru, szansa, że odgadnie prawidłową odpowiedź wynosi $\frac{1}{5}$. ~ \bigskip
		\\Jakie jest prawdopodobieństwo warunkowe tego, że student znał odpowiedź na dane pytanie, jeśli nie popełnił w tym pytaniu błędu?
	}

	\frame{
		\frametitle{Definicje}
		\fontsize{14pt}{7.2}\selectfont
		 \large Twierdzenie Bayesa to twierdzenie teorii prawdopodobieństwa, wiążące prawdopodobieństwa warunkowe dwóch zdarzeń warunkujących się nawzajem, sformułowane przez Thomasa Bayesa.\\
	}

	\frame{
	\frametitle{Definicje}
	\fontsize{14pt}{7.2}\selectfont		
	\large W podstawowej formie twierdzenie Bayesa mówi, że: \\
	$P(A_{i} | B) = \dfrac{P(B|A_{i})P(A_{i})}{P(B)}$, gdzie: $P(B) = \sum_{i \in j} P(B|A_{j}) \cdot P(A_{j}) $ \\
	\bigskip Gdzie $A$ i $B$ są zdarzeniami oraz $P(B) > 0$, przy czym:
	
	\begin{itemize}
		\item $P(A|B)$ oznacza prawdopodobieństwo warunkowe (prawdopodobieństwo zajścia zdarzenia A, o ile zajdzie zdarzenie B).~
		\item $P(B|A)$ oznacza prawdopodobieństwo zajścia zdarzenia B, o ile zajdzie zdarzenie A.~
	\end{itemize}~
	}
		
	\frame{
	\frametitle{Rozwiązanie}
	\fontsize{14pt}{7.2}\selectfont	
	
	Z treści zadania można odczytać następujące dane:
	\begin{itemize}
		\item $A_{1}$ - uczeń zna odpowiedź,
		\item $A_{2}$ -  uczeń nie zna odpowiedzi,
		\item $B$ - uczeń nie popełnił błędu.
	\end{itemize}
	
	\bigskip Podstawiając dane do wzoru Bayesa: \\
	\bigskip $P(A_{1}|B) = \dfrac{P(B|A_{1}) \cdot P(A_{1})}{P(A_{1}) \cdot P(B|A_{1}) + P(A_{2}) \cdot P(B|A_{2})}$
	}


	\frame{
		\frametitle{Rozwiązanie}
		\fontsize{15pt}{7.2}\selectfont		
		
		Z treści zadania wynika:
		\begin{itemize}
			\item $P(A_{1}) = P(A_{2}) = \frac{1}{2}$, 
			\item $P(B|A_{1}) = 1$,
			\item $P(B|A_{2}) = \frac{1}{5}$.
		\end{itemize}
		\bigskip $P(A_{1}|B) = \dfrac{\frac{1}{2} \cdot 1}{\frac{1}{2} \cdot 1 + \frac{1}{2} \cdot \frac{1}{5}}$\\
				
	}

	\frame{
	\frametitle{Rozwiązanie}
	\fontsize{15pt}{7.2}\selectfont	
	\bigskip $P(A_{1}|B) = \dfrac{1}{2(\frac{\frac{1}{5}}{2} + \frac{1}{2})} = \dfrac{1}{2(\frac{1}{10} + \frac{1}{2})}$\\
	\bigskip $P(A_{1}|B) = \dfrac{1}{2\cdot \frac{5+1}{10}} = \dfrac{1}{2 \cdot \frac{6}{10}}$\\
	\bigskip $P(A_{1}|B) = \dfrac{\dfrac{1}{2}}{\dfrac{3}{5}}$\\
	
	
	}

	\frame{
	\frametitle{Rozwiązanie}
	\fontsize{15pt}{7.2}\selectfont	

	\bigskip $P(A_{1}|B) = \dfrac{5}{2 \cdot 3}$ \\
	\bigskip $P(A_{1}|B) = \dfrac{5}{6}$	
	}

	\frame{
	\frametitle{Odpowiedź}
	\fontsize{15pt}{7.2}\selectfont			
	Prawdopodobieństwo warunkowe tego, że student znał odpowiedź na pytanie, jeśli nie popełnił błędu, wynosi $\frac{5}{6}$.
	}

		
	\frame{
		\centering \Huge
		\emph{Dziękuję za uwagę!}
	}
	
\end{document}
