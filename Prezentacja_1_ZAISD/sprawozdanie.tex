\documentclass[UTF8]{article}

\usepackage{amsmath}
\usepackage[utf8]{inputenc}
\usepackage[OT4]{polski}
\usepackage{graphicx}
\usepackage{subcaption}
\usepackage{geometry}

\geometry{left=2.5cm,right=2.5cm,top=2.5cm,bottom=2.5cm}

\title{%
	Sprawozdanie 1 \\
	\large Zadanie 24. Przedstaw liczbę 100 jako sumę dwóch liczb całkowitych dodatnich, których iloczyn jest maksymalny.~}

\author{Adrian Rupala}


\begin{document}
\pagenumbering{gobble}
\maketitle

\newpage
\tableofcontents

\newpage
\pagenumbering{arabic}

\section{Teoria}



\section{Rozwiązanie}

Z treści zadania wynika, że potrzebujemy zdefiniować dwie liczby naturalna dodanie, w przypadku tego rozwiązania zostały one zdefiniowane jako x oraz y. Ich suma musi być równa 100. Jesteśmy również w stanie za pomocą powstałego równania wyznaczyć jedną z wartości niewiadomych. ~
\\$x + y = 100 \Rightarrow y = x - 100 $\\
\\Kolejnym krokiem jest utworzenie funkcji która będzie opisywała stosunek dwóch liczb których iloczyn jest maksymalny, podstawienie do niego obliczonej wartości $y$ którą obliczyliśmy przekształcając powyższe równanie.~
\\$ f(x) = x \cdot y = x \cdot (100-x) = 100x - x^{2} $\\
\\Następnie aby otrzymać wynik należy obliczyć pochodną funkcji pierwszego stopnia. Otrzymane w ten sposób równanie przyrównujemy do zera. Obliczamy pochodną funkcji  ponieważ chcemy określić konkretny punkt w jakim nasz wykres paraboli przecinałby w maksymalnym punkcie oś $ x $.~
\\$ \frac{\partial}{\partial x}(100x - x^{2}) = 0$
\\$100 - 2x = 0$
\\$2x = 100$
\\$x = 50 $\\
\\W ten sposób obliczyliśmy wartość pierwszej szukanej przez nas liczby. Aby obliczyć kolejną wystarczy obliczoną liczbę wstawić do pierwszego wzoru opisującego stosunek dwóch liczb do liczby jaką chcemy przedstawić.~
\\$ y = 50 - 100 \Rightarrow y = 50 $\\
\\W ten sposób obliczyliśmy poszukiwane przez nas dwie maksymalne liczby naturalne których iloczyn jest maksymalny.~

\section{Odpowiedź}

	Szukane przez nas dwie wartości dla liczby 100 to x = 50 oraz y = 50. ~

\end{document}