\documentclass[10pt,aspectratio=169]{beamer}
\usetheme{Madrid}

\usepackage[utf8]{inputenc}
\usepackage[OT4]{polski}
\usepackage{flexisym}
\usepackage{parskip}


\title{Zadanie 24. Maksymalny iloczyn}
\subtitle{Adrian Rupala}


\begin{document}
	\frame {
		\titlepage
	}

	\frame {
		\frametitle{Treść zadania}
		Przedstaw liczbę 100 jako sumę dwóch liczb całkowitych dodatnich, których iloczyn jest maksymalny.~
	}

	\frame{
		\frametitle{Typ zadania}
		\framesubtitle{Z czym to się je?}
		\begin{itemize}
			\item Jest to zadanie optymalizacyjne: celem jest znalezienie ekstremum funkcji.~
			\item Należy definiować liczby jako składniki, z czego jeden parametr musi być zależny od drugiego.~
			\item Następnie układamy równanie zgodnie z treścią zadania.~
			\item Kolejnym krokiem jest obliczenie pochodnej funkcji pierwszego rzędu dla wcześniej utworzonego równania.~
			\item Wynikiem pochodnej jest jedna z naszych wartości, drugą można obliczyć podstawiając do wcześniejszego wzoru, który utworzyliśmy z treści zadania.~  
		\end{itemize}

	}

	\frame{
		\frametitle{Rozwiązanie}
				
		\begin{align*}
			\text{Z treści zadania wynika: ~}\\
			x + y &= 100 \Rightarrow y = 100 - x \\
			\text{Wykorzystując maksymalizację: ~}\\
			f(z) &= x \cdot y = x \cdot (100-x) = 100x - x^{2} \\
			\text{Pochodną funkcji przyrównujemy do zera: ~}\\
			\frac{\partial}{\partial x}(100x - x^{2}) &= 0\\
			100 - 2x &= 0\\
			2x &= 100\\
			x &= 50 \\
			\text{Zgodnie z zadaniem y \textgreater 0, więc: ~}\\
			y &= 50\\
			x+y &= 100 \Rightarrow 50 + 50 = 100
		\end{align*}
	}
	
	\frame{
		\centering \Huge
		\emph{Dziękuję za uwagę!}
	}
	
\end{document}
