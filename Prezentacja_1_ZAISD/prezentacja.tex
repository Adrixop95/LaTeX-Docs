\documentclass[aspectratio=169]{beamer}
\usetheme{metropolis}

\usepackage[utf8]{inputenc}
\usepackage[OT4]{polski}
\usepackage{flexisym}
\usepackage{parskip}


\title{Zadanie 24. Maksymalny iloczyn}
\subtitle{Adrian Rupala}


\begin{document}
	\frame {
		\titlepage
	}

	\frame {
		\frametitle{Treść zadania}
		Przedstaw liczbę 100 jako sumę dwóch liczb całkowitych dodatnich, których iloczyn jest maksymalny.~
	}

	\frame{
		\frametitle{Typ zadania}
		\begin{itemize}
			\item Jest to zadanie optymalizacyjne; celem jest znalezienie ekstremum funkcji.~
			\item Należy definiować liczby jako składniki, z czego jeden parametr musi być zależny od drugiego.~
			\item Układamy równanie zgodnie z treścią zadania.~
			\item Obliczamy pochodnej funkcji pierwszego rzędu dla wcześniej utworzonego równania.~
			\item Wynikiem ekstremum jest jedna z naszych wartości, drugą można obliczyć podstawiając do wcześniejszego wzoru, który utworzyliśmy z treści zadania.~  
		\end{itemize}

	}

	\frame{
		\frametitle{Rozwiązanie}
				
		\begin{align*}
			\text{Z treści zadania wynika: ~}\\
			x + y &= 100 \Rightarrow y = 100 - x \\
			\text{Tworzymy funkcję zgodnie z treścią: ~}\\
			f(x) &= x \cdot y = x \cdot (100-x) = 100x - x^{2} \\
			\text{Obliczamy pochodną oraz ekstremum: ~}\\
			\frac{\partial}{\partial x}(100x - x^{2}) &= 0\\
			100 - 2x &= 0 \Rightarrow 2x = 100 \Rightarrow x = 50\\
			\text{Obliczanie wartości $y$: ~}\\
			y &= 100 - 50 \Rightarrow y = 50
		\end{align*}
	}
	
	\frame{
		\centering \Huge
		\emph{Dziękuję za uwagę!}
	}
	
\end{document}
