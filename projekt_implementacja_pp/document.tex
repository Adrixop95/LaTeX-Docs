\documentclass[15pt, aspectratio=169]{beamer}
\usetheme{metropolis}

\usepackage[utf8]{inputenc}
\usepackage[OT4]{polski}
\usepackage{flexisym}
\usepackage{parskip}


\title{Optymalizacja kodu źródłowego - porównanie czasów działania}
\subtitle{Adrian Rupala}


\begin{document}
	\frame {
		\titlepage
	}

	\frame {
		\frametitle{Plan prezentacji}
		\begin{itemize}
			\item Co to jest optymalizacja? ~\\
			\item Rodzaje optymalizacji. ~\\
			\item Jak uzyskać optymalny kod? ~\\
			\item Platformy oraz ograniczenia. ~\\
			\item Przykłady kodu. ~\\
		\end{itemize}
	}

	\frame{
	\frametitle{Co to jest optymalizacja?}
		\begin{itemize}
			\item Działania mające na celu poprawę wydajności programu. ~
			\item Zwiększanie szybkości wykonania kodu poprzez zmniejszenie zapotrzebowania na zasoby. ~
			\item Poprawienie algorytmu w celu usunięcia, poprawy, naprawy błędów. ~
			\item Dostosowanie programu do wykonania konkretnego zadania. ~
		\end{itemize}	
	}

	
	\frame{
	\frametitle{Rodzaje optymalizacji}
		\begin{itemize}
			\item Optymalizacja projektu ~
			\item Optymalizacja algorytmów i struktur danych ~
			\item Optymalizacja kodu źródłowego ~
			\item Optymalizacja budowy projektu ~
			\item Optymalizacja kompilowania ~
			\item Optymalizacja składania projektu ~
			\item Optymalizacja uruchamiania ~
				
		\end{itemize}
	}

	\frame{
	\frametitle{Jak uzyskać optymalny kod?}	

	}

	\frame{
		\frametitle{Platformy oraz ograniczenia}
	}

	\frame{
		\frametitle{Przykłady kodu}
	}

	\frame{
		\centering \Huge
		\emph{Dziękuję za uwagę!\\}
		\emph{Czy są jakieś pytania?}
	}
	
\end{document}
